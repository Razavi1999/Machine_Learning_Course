\begin{boxK}
    \begin{equation*}
        P(Minus | Doc1) < P(Plus | Doc1)
    \end{equation*}

    \begin{equation*}
        P(Minus | Doc2) > P(Plus | Doc2)
    \end{equation*}

    \begin{equation*}
        P(Minus | Doc3) < P(Plus | Doc3)
    \end{equation*}

    \begin{equation*}
        P(Minus | Doc4) > P(Plus | Doc4)
    \end{equation*}

    \begin{equation*}
        P(Minus | Doc5) < P(Plus | Doc5)
    \end{equation*}

\end{boxK}


\begin{boxC}
    با توجه به نتایج بالا ، هر یک از اسناد برچسب متناظر خود را دریافت خواهند کرد.

    بر اساس منطق این رده‌بند ، هر کدام که از این اسناد که احتمال شرطی بالاتری برای هر کدام از کلاس‌ها داشته‌باشد ، به آن کلاس مربوط می‌شود.

    باید به مقایسه دو عبارت زیر به ازای سند ۶ و ۷ بپردازیم و سپس برچسب متناظر با آن داده را بر روی سند مدنظر بزنیم.

    \begin{equation*}
        P(plus) * P(W_{1} | plus) * P(W_{2} | plus) * P(W_{3} | plus) * ... * P(W_{n} | plus)
    \end{equation*}

      \begin{equation*}
        P(minus) * P(W_{1} | minus) * P(W_{2} | minus) * P(W_{3} | minus) * ... * P(W_{n} | minus)
    \end{equation*}

    با توجه به معادلات جایگاه نامعادلات بالا ، به داده‌های زیر خواهیم رسید : 

    \begin{equation*}
        P(plus) = \frac{3}{5}
    \end{equation*}

    \begin{equation*}
        P(minus) = \frac{2}{5}
    \end{equation*}
    
\end{boxC}

\begin{boxK}
    \begin{equation*}
        P(love | plus) = \frac{1 + 1}{11 + 6} = \frac{2}{17}  
    \end{equation*}

    \begin{equation*}
        P(movie | plus) = \frac{4 + 1}{11 + 6} = \frac{5}{17}  
    \end{equation*}

    \begin{equation*}
        P(great | plus) = \frac{2 + 1}{11 + 6}  = \frac{3}{17}  
    \end{equation*}

    \begin{equation*}
        P(good | plus) = \frac{2 + 1}{11 + 6} = \frac{3}{17}
    \end{equation*}

    \begin{equation*}
        P(acting | plus) = \frac{1 + 1}{11 + 6} = \frac{2}{17}    
    \end{equation*}

    \begin{equation*}
        P(I | plus) = \frac{1 + 1}{11 + 6} = \frac{2}{17}   
    \end{equation*}
    
\end{boxK}

\begin{boxK}
    \begin{equation*}
        P(hated | minus) = \frac{1 + 1}{5 + 5} = \frac{2}{10}
    \end{equation*}

    \begin{equation*}
        P(I | minus) = \frac{1 + 1}{5 + 5} = \frac{2}{10}
    \end{equation*}

    \begin{equation*}
        P(movie | minus) = \frac{1 + 1}{5 + 5} = \frac{2}{10}
    \end{equation*}

    \begin{equation*}
        P(poor | minus) = \frac{1 + 1}{5 + 5} = \frac{2}{10}
    \end{equation*}

    \begin{equation*}
        P(acting | minus) = \frac{1 + 1}{5 + 5} = \frac{2}{10}
    \end{equation*}
\end{boxK}

\begin{boxC}
    \begin{center}
        \lr{Document6 : I loved the poor play.}
        % \newline
    \end{center}

    \begin{equation*}
        P(plus | Document6) = P(plus) * \frac{P(love | plus) * P(I | plus) * P(poor | plus) * P(play | plus)}{P(Document6)}
    \end{equation*}

    \begin{equation*}
        P(minus | Document6) = P(minus) * \frac{P(love | minus) * P(I | minus) * P(poor | minus) * P(play | minus)}{P(Document6)}
    \end{equation*}

    با توجه به این که مقدار احتمال سند ششم برای هر دو ترم با هم برابر است ، به صورت زیر تخمین خواهیم زد :
    
    \begin{equation*}
        P(plus | Document6) = \frac{3}{5} * \frac{2}{17} * \frac{2}{17} * \frac{1}{17} * \frac{1}{17} 
        = 0.000028735 
    \end{equation*}

    \begin{equation*}
        P(minus | Document6) = \frac{2}{5} * \frac{1}{10} * \frac{2}{10} * \frac{2}{10} * \frac{1}{10} = 
        0.00016
    \end{equation*}

    برچسب سند ششم منفی خواهد بود.
\end{boxC}

\newpage

\begin{boxC}
    \begin{center}
        \lr{Document7 : ‫‪I‬‬ ‫‪hated‬‬ ‫‪the‬‬ ‫‪play‬‬ ‫‪movie‬‬.}
    \end{center}

    همانند جایگاه معادلات بالا برای سند ۶ ، مقادیر احتمالاتی را برای کلاس‌های ممکن برای سند ۷ را محاسبه خواهیم کرد :

    \begin{equation*}
        P(plus | Document7) = \frac{3}{5} * \frac{2}{17} * \frac{1}{17} * \frac{1}{17} * \frac{5}{17} = 0.00007183822
    \end{equation*}

    \begin{equation*}
        P(minus | Document7) = \frac{2}{5} * \frac{2}{10} * \frac{2}{10} * \frac{2}{10} * \frac{1}{10} = 0.00032
    \end{equation*}

    
    برچسب سند هفتم منفی خواهد بود.    
\end{boxC}

\begin{boxL}
    با توجه به نتایج بالا و وزن  کلمه مثبتی همچون 
    \lr{love}
    و وزن منفی کلمه‌ای چون 
    \lr{poor}
    می‌توان پیش‌بینی نمود که وزن بیشتر به سمت کلاس منفی باشد ، چرا که احتمال منفی به شرط کلمه 
    \lr{poor}
    بسیار بیشتر از احتمال مثبت به شرط کلمه
    \lr{love}
    است.
    
    \begin{equation*}
        P(minus | poor) > P(plus | love)
    \end{equation*}
    
    
    و همچنین وزن کلمه منفی مانند
    \lr{hate}
    همچین نتایجی قابل پیش‌بینی بود.

    نتایج به دست‌آمده تا حد خوبی منطقی هستند. 

     
     این نتایج به نظر بیشتر وابسته به داده‌های 
    پس‌زمینه 
    \lr{Background}
    ما هستند.
    وابستگی میزان احتمال هر کلاس به شرط هر کلمه که از داده‌های پس‌زمینه ما به دست آمدند این قضیه را بیشتر اثبات می‌کند.
    
\end{boxL}

