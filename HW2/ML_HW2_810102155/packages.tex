\usepackage[]{algorithm2e}
\usepackage{subfig}

\usepackage{cite}
\usepackage{calc}
\usepackage{minted}
%\usepackage{fancyhdr}
\usepackage{color}
\usepackage{ragged2e}
\usepackage[inline]{enumitem}
\usepackage[dvipsnames]{xcolor}
\usepackage{graphicx}
\usepackage{wrapfig}
\usepackage{float}
\usepackage{hyperref}
\usepackage{lipsum}
\usepackage{tabu}
\usepackage{hyperref}
\usepackage{minted}
\usepackage{listings}
% \usepackage{background}
% \AddToHook{shipout/background}{\put(0pt, -\paperheight) 
% {\includegraphics[width =\paperwidth, height =\paperheight]{images/iran.jpg}}}



\usepackage[margin=0.5in]{geometry} % for PAPER & MARGIN
\usepackage[many]{tcolorbox}    	% for COLORED BOXES (tikz and xcolor included)
\usepackage{mathspec} 			    % for FONTS
\usepackage{setspace}               % for LINE SPACING
\usepackage{multicol} 
\usepackage{xepersian}


\lstdefinestyle{customc}{
  belowcaptionskip=1\baselineskip,
  breaklines=true,
  frame=L,
  xleftmargin=\parindent,
  language=C,
  showstringspaces=false,
  basicstyle=\footnotesize\ttfamily,
  keywordstyle=\bfseries\color{green!40!black},
  commentstyle=\itshape\color{purple!40!black},
  identifierstyle=\color{blue},
  stringstyle=\color{orange},
}

\lstset{
    style=customc
    frame=tb,
    language=Python,
    aboveskip=3mm,
    belowskip=3mm,
    showstringspaces=true,
    columns=flexible,
    basicstyle={\small\ttfamily},
    numbers=none,
    numberstyle=\tiny\color{gray},
    keywordstyle=\color{blue},
    commentstyle=\color{dkgreen},
    stringstyle=\color{mauve},
    breaklines=true,
    breakatwhitespace=true,
    tabsize=3
    }

\settextfont[BoldFont={XB Yas Bd}, ItalicFont={XB Yas It}, BoldItalicFont={XB Yas BdIt}, Extension = .ttf]{XB Yas}


\fancyhead[L]{\includegraphics[scale = 0.02]{multiAgent/images/utlogo.jpeg}}
\fancyhead[R]{\large \Course}
\fancyhead[C]{\Large \Subject}
\renewcommand{\headrulewidth}{0.5mm}
\pagestyle{fancy}
\setlength{\headheight}{28pt}

\setminted{
    frame=lines,
    framesep=2mm,
    baselinestretch=1.0,
    xrightmargin=0in,
    xleftmargin=0in,
    fontsize=\footnotesize,
    linenos=false
}

\newcommand{\grayBox}[1]{\colorbox{gray!10}{\lr{\texttt{#1}}}}
\newcommand{\grayLBox}[1]{\fbox{\grayBox{\parbox{11.8 cm}{#1}}}}

\makeatletter
\newcommand{\chapterauthor}[1]{%
    {\parindent0pt\vspace*{-25pt}%
        \linespread{1.1}\large\scshape#1%
        \par\nobreak\vspace*{35pt}}
    \@afterheading%
}
\makeatother

% \documentclass[10pt]{book}
              % for MULTICOLUMNS

% \setmainfont{Noto Sans}[
%     Kerning = On,
%     Mapping = tex-text,
%     Numbers = Uppercase, 
%     BoldFont = Noto Sans SemiBold
% ]                           % setting the font as Noto Sans
\setlength\parindent{0pt}   % killing indentation for all the text
\setstretch{1.3}            % setting line spacing to 1.3
\setlength\columnsep{0.25in} % setting length of column separator
% \pagestyle{empty}           % setting pagestyle to be empty


\definecolor{main}{HTML}{5989cf}    % setting main color to be used
\definecolor{sub}{HTML}{cde4ff}     % setting sub color to be used

\tcbset{
    sharp corners,
    colback = white,
    before skip = 0.2cm,    % add extra space before the box
    after skip = 0.5cm      % add extra space after the box
}                           % setting global options for tcolorbox

% You can copy any following box you like to your code.
\newtcolorbox{boxA}{
    fontupper = \bf,
    boxrule = 1.5pt,
    colframe = black % frame color
}

\newtcolorbox{boxB}{
    fontupper = \bf\color{main}, % font color
    boxrule = 1.5pt,
    colframe = main,
    rounded corners,
    arc = 5pt   % corners roundness
}

\newtcolorbox{boxC}{
    colback = sub, % background color
    boxrule = 0pt  % no borders
}

\newtcolorbox{boxD}{
    colback = sub, 
    colframe = main, 
    boxrule = 0pt, 
    toprule = 3pt, % top rule weight
    bottomrule = 3pt % bottom rule weight
}

\newtcolorbox{boxE}{
    enhanced, % for a fancier setting,
    boxrule = 0pt, % clearing the default rule
    borderline = {0.75pt}{0pt}{main}, % outer line
    borderline = {0.75pt}{2pt}{sub} % inner line
}

\newtcolorbox{boxF}{
    colback = sub,
    enhanced,
    boxrule = 1.5pt, 
    colframe = white, % making the base for dash line
    borderline = {1.5pt}{0pt}{main, dashed} % add "dashed" for dashed line
}

\newtcolorbox{boxG}{
    enhanced,
    boxrule = 0pt,
    colback = sub,
    borderline west = {1pt}{0pt}{main}, 
    borderline west = {0.75pt}{2pt}{main}, 
    borderline east = {1pt}{0pt}{main}, 
    borderline east = {0.75pt}{2pt}{main}
}

\newtcolorbox{boxH}{
    colback = sub, 
    colframe = main, 
    boxrule = 0pt, 
    leftrule = 6pt % left rule weight
}

\newtcolorbox{boxI}{
    colback = sub, 
    colframe = main, 
    boxrule = 0pt, 
    toprule = 6pt % top rule weight
}

\newtcolorbox{boxJ}{
    sharpish corners, % better drop shadow
    colback = sub, 
    colframe = main, 
    boxrule = 0pt, 
    toprule = 4.5pt, % top rule weight
    enhanced,
    fuzzy shadow = {0pt}{-2pt}{-0.5pt}{0.5pt}{black!35} % {xshift}{yshift}{offset}{step}{options} 
}

\newtcolorbox{boxK}{
    sharpish corners, % better drop shadow
    boxrule = 0pt,
    toprule = 4.5pt, % top rule weight
    enhanced,
    fuzzy shadow = {0pt}{-2pt}{-0.5pt}{0.5pt}{black!35} % {xshift}{yshift}{offset}{step}{options} 
}

\newtcolorbox{boxL}{
    fontupper = \color{main},
    rounded corners,
    arc = 6pt,
    colback = sub, 
    colframe = main!50, 
    boxrule = 0pt, 
    bottomrule = 4.5pt 
}

\newtcolorbox{boxM}{
    fontupper = \color{white},
    rounded corners,
    arc = 6pt,
    colback = main!80, 
    colframe = main, 
    boxrule = 0pt, 
    bottomrule = 4.5pt,
    enhanced,
    fuzzy shadow = {0pt}{-3pt}{-0.5pt}{0.5pt}{black!35}
}
